\documentclass[paper=letter, fontsize=10pt]{scrartcl}

\usepackage{geometry}
\usepackage[T1]{fontenc}
\usepackage[utf8]{inputenc}
\usepackage[english, spanish]{babel}
\usepackage{hyperref}
%% \usepackage[superscript,biblabel]{cite}
\usepackage[backend=bibtex, style=trad-abbrv]{biblatex}
\usepackage{sectsty}

\addbibresource{bibliography.bib}

\geometry{
  top=1cm,
  right=2cm,
  left=2cm,
  bottom=1cm
}

\allsectionsfont{\raggedright\large \textit\normalfont\scshape\emph}

\title{
  \normalsize
  UNIVERSIDAD NACIONAL AUTÓNOMA DE MÉXICO\\
  Facultad de Ciencias\\
  Licenciatura en Ciencias de la Computación
}

\subtitle{\normalsize Propuesta de Tesis de Licenciatura}

\author{\normalsize Albert Manuel Orozco Camacho}

\date{\normalsize \today}

\begin{document}

\maketitle

\section*{Título}

\noindent
\emph{\textbf{Generación automática de memes de Internet a través de una red neuronal profunda}}

\section*{Director de Tesis}

\thispagestyle{empty}
\noindent
Dr. Ivan Vladimir Meza Ruiz\\
Departamento de Ciencias de la Computación\\
Instituto de Investigaciones en Matemáticas Aplicadas y en Sistemas (\emph{IIMAS})\\
Universidad Nacional Autónoma de México (\emph{UNAM})\\
\texttt{\href{mailto:ivanvladimir@turing.iimas.unam.mx}{ivanvladimir@turing.iimas.unam.mx}}

\section*{Introducción}

\noindent
Se le llama \emph{\textbf{meme de Internet}} a la unidad de propagación de información vía electrónica,\
que muchas veces utiliza bromas, chistes y/o rumores en su estructura\cite{shifman2014}.\
La exorbitante popularidad del Internet ha permitido el crecimiento del uso de memes,\
en particular, con el formato de imagen y leyenda corta. El problema de la generación automática de\
memes de Internet ha sido escasamente estudiado dentro de las disciplinas del procesamiento de\
lenguaje natural y visión computacional.\par
En este trabajo, se abordará dicho problema mediante el uso de aprendizaje profundo, una rama\
del aprendizaje automático que ha redefinido el \emph{estado del arte} de diversas tareas en inteligencia artificial.\
Se recolectará una gran cantidad de memes de Internet, separando la imagen de su leyenda y priorizando\
el modelo que predomina en el sitio web \texttt{MemeGenerator.net}\footnote{\url{https://memegenerator.net/}}.\
Partiendo de un conjunto de $100$ imágenes con $1500$ leyendas cada una, equivalentes a $5$ GB de información,\
se entrenará una red neuronal profunda. Para la implementación, se utilizarán primordialmente las bibliotecas\
\texttt{Tensorflow}\cite{tensorflow2015-whitepaper} y \texttt{Keras}\cite{chollet2015keras} que funcionan\
para el lenguaje de programación Python.\par
El desempeño de la misma será evaluado por su capacidad de generar una leyenda para una imagen no perteneciente al conjunto de\
datos de entrenamiento y qué tanto \emph{sentido} tiene el meme generado desde una perspectiva empírica.\par
El uso de memes como vía de comunicación ha incrementado durante los últimos años, tanto así que ya es\
parte del recurso periodístico que diversos medios ofrecen. Un sistema de generación automática de memes\
promete ser de gran utilidad en una era en la cual la información fluye de manera rápida y breve, a través\
de las redes sociales.

\section*{Objetivos}

\noindent
Diseñar e implementar un sistema computacional que sea capaz de extraer las características más importantes\
que relacionan una imagen con su leyenda en un meme, utilizando herramientas de aprendizaje profundo. A partir\
de lo anterior, dada una imagen desconocida por el sistema, producir la leyenda que más haga sentido para formar\
un meme, con los conceptos (características) aprendidos en el primer paso.

\section*{Metas}

\begin{itemize}
\item Recolectar una importante cantidad de memes de internet, clasificándolos por idioma.\
  La mayor parte de dicho conjunto de datos será para entrenar a la red neuronal que se desarrollará,\
  ¿mientras que la otra será para evaluar el desempeño de la misma?.
\item Extraer la leyenda (texto) de cada imagen, de manera que ambas partes queden separadas\
  pero sin perder su asociación.
\item Diseñar e implementar la red neuronal profunda que extraerá las principales características de cada imagen\
  y su leyenda asociada.
\item Implementar un mecanismo para digitalizar las imágenes y que la red neuronal sea capaz de trabajar\
  con éstas y sus leyendas asociadas.
\item Entrenar la red neuronal con el conjunto de datos recolectado de Internet.
\item Evaluar el desempeño de la salida de la red, asociando imagen con leyenda.
\end{itemize}

\section*{Estructura de la Tesis}

\begin{enumerate}
\item Introducción
\item Marco teórico
\item Red neuronal para descripciones de memes
\item Evaluación del desempeño de la red
\item Conclusiones
\end{enumerate}

\nocite{*}
\printbibliography[title={Propuesta de bibliografía}]

%% \begin{thebibliography}{99}
%% \bibitem{Shifman} Shifman, L. (2013), Memes in a Digital World: Reconciling with a Conceptual Troublemaker.\
%%   J Comput-Mediat Comm, 18: 362–377. doi:10.1111/jcc4.12013
%% \end{thebibliography}
\end{document}
